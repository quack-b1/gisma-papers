\chapter{Quality Management}
\section{Quality Objectives and Standards}

\parindent20pt The quality objectives of the CHSR project emphasize three primary outcomes: compliance with federal and state regulatory agencies regarding safety and performance; excellence in design and construction that ensures durability, requires minimal maintenance, and exhibits resilience; and adherence to standards for customer experience, including timeliness, ride comfort, and accessibility. The CHSR authority’s Quality Management System (QMS) is based on ISO 9001:2015 \citep{sustainability2024}, which integrates requirements for contractor quality assurance (QA) and quality control (QC).

\section{Quality Assurance and Quality Control Framework}
The authority maintains a comprehensive, multi-level QA/QC framework that includes contractors responsible for initial quality control, such as material testing and inspection; independent quality assurance firms (IQFs) hired to verify contractor results; and supervision by the CHSR authority to ensure compliance with standards throughout the entire project. Quality is monitored at every stage— from design validation and environmental compliance to civil construction and system integration.

\section{Challenges in Quality Management}
Despite its robust architecture, the CHSR has encountered multiple challenges related to quality management: \begin{itemize}
\item Design modifications and initial engineering complications, resulting in rework and increased expenses.
\item Inconsistent quality among contractors, particularly in subcontracted earthworks and viaduct segments.
\item Delays in coordinating with third-party utility providers, adversely affecting inspections and certifications. \end{itemize}
These challenges have necessitated the adoption of a more integrated quality governance framework that enhances coordination among IQFs, contractors, and CHSRA engineering teams.\par