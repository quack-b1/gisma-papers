\chapter{Project Scope}
\section{Initial Plan}
The California High-Speed Rail (CHSR) represents one of the few voter-approved initiatives at the state level. This high-speed rail project adheres to elevated environmental standards, intending to develop infrastructure, railways, and high-speed trains across California. Notable features of these railway systems include trains capable of achieving speeds of up to 220 miles per hour (354 kmph), operation utilizing 100\% renewable energy, and seamless integration with modern transportation systems. The project is delineated into two principal phases.

The first phase of the initial operational railway extends from Merced to Bakersfield in the Central Valley. Ultimately, this phase will be considered complete when the rail line connects San Francisco to Los Angeles/Anaheim, covering an approximate distance of 500 miles (805 km) with a total travel time of under three hours. The second phase aims to extend the initial line northward to Sacramento and southward to San Diego, culminating in a comprehensive railway network spanning 800 miles. A critical component of quality control within this project is the project management team's dedication to producing periodic reports and ensuring transparency throughout the project's scope.

\section{Current vs Initial Plans}

