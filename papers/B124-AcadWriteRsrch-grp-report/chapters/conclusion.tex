\chapter{Conclusion}

This paper examines whether Berlin’s reputation as an international, English-friendly city matches the linguistic requirements of its job market for non-German professionals. Results show a gap between perception and reality: proficiency in German remains a key, often mandatory, requirement across many industries. Despite Berlin’s marketing as multicultural and English-friendly, with minimal German required, most job listings are not in English, and German fluency is often a prerequisite. This challenges the notion that English proficiency alone is sufficient for employment outside niche sectors, such as expatriate tech startups. 

Additionally, the survey results highlight this mismatch, with 88\% of respondents reporting job application rejections due to limited German skills. Only 12\% found employment without fluency in German, mainly in the hospitality and IT sectors. This supports the idea that language proficiency is strongly linked to better job market outcomes.

The study reveals a gap between the expectations of non-German-speaking professionals before relocating to Berlin and their actual job market experiences. Three of the four hypotheses were confirmed, with the fourth—regarding students and early-career professionals facing more language barriers—left for future research due to space limitations. Overall, German language skills are a significant barrier to entry, underscoring the need for a clearer understanding of the linguistic demands in Berlin’s job market.

These findings suggest that policymakers and city branding should better align Berlin’s international image with its labor practices. Improving communication about language expectations can help manage talent’s expectations. Universities and career services may enhance German language support to prepare international students for the workforce. Employers seeking diverse talent may want to reconsider their German-language requirements or provide in-house training. Future research could focus on sector-specific analyses or employer perspectives for a more detailed understanding of language-based hiring barriers in Berlin.
