\chapter{Conclusion}

This paper aimed to investigate the extent to which Berlin’s international and English-friendly reputation aligns with the linguistic requirements of its labor market for non-German-speaking professionals. The findings reveal a significant disparity between the city’s perceived image and the reality of its employment landscape. The study consistently demonstrates that proficiency in the German language remains a crucial, often mandatory requirement across a wide range of industries in Berlin. While Berlin is marketed as an exemplary, multicultural, and English-friendly city that requires minimal German language skills, the data indicate that a substantial majority of job listings are not exclusively in English, and German fluency is frequently needed. This directly challenges the perception that proficiency in English is sufficient for securing employment outside a narrow niche of expatriate-based technology startups. 

Additionally, the survey findings further underscore this misalignment, with a significant 88\% of respondents reporting job application rejections due to insufficient German language skills. Only 12\% managed to secure employment without fluent German, primarily within the hospitality and information technology sectors. This evidence supports the assertion that proficiency in the local language has a strong correlation with improved market results.

Furthermore, the study reveals a notable gap between the expectations of non-German-speaking professionals before relocating to Berlin and their actual experiences in the job market. Three of the study’s four hypotheses have been successfully proven correct, with the fourth hypothesis, regarding students and early-career professionals facing greater language-related employment barriers, left for future exploration due to the constraints of this writing. In conclusion, German language proficiency serves as a significant barrier to entry, underscoring the need for a more realistic portrayal of the linguistic demands of Berlin’s job market.
