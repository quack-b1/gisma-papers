\chapter{Methodology}

To address this study’s research question, this paper employs a mixed-method approach comprising primary and secondary research strategies. This will provide a comprehensive perspective on how language impacts employability in Berlin, especially for individuals who are not fluent in German. Additionally, this mixed approach will allow for a relative comparison of secondary works and current-day testimonials as of the summer of 2025.

Primary data was collected via an online survey distributed through university mailing lists, online forums, and social media platforms. The survey was created using Microsoft Office Forms (see Appendix A). It targeted expatriates and international students and was conducted anonymously with informed consent provided through a data protection disclaimer (datenschutzhinweis) written in both German and English. Additionally, the survey includes sections on demographics, academic background, employment preferences, language skills, and job search experience. Thus, the survey enables the collection of both statistical trends and subjective perceptions regarding the role of language in Berlin’s job market.

Primary research aids and benefits this paper by ensuring contemporary data is included. The survey used in this paper allows this research to “avoid unreliable information from outside sources” (Indeed, 2025). However, primary research comes with its respective challenges (Karunarathna et al., 2024). The survey conducted in this study was particularly time-consuming and difficult to distribute. Issues arose when convincing individuals to take a few minutes to complete the survey, as many individuals had refused to participate in such a study without an incentive. Also, the sample size was limited due to the difficulty in distributing the survey to the large number of Berlin residents. Some respondents may also downplay the issue of language requirements if they had help from inside their company, or, contrary, where some may exaggerate the issue due to bad experiences during the hiring process.

Secondary data was collected through a content analysis of job listings on platforms like LinkedIn, Indeed, StepStone, and businesses’ direct hiring portals. Said listings include jobs from various industries, including technology, marketing, education, healthcare, customer service, government, and finance. Each job listing was analyzed to determine the language it was posted in, the language requirements listed, the industry, and the industry level. This data helps evaluate how frequently English-friendly positions are advertised, which industries offer such positions, and whether German language skills are a formal prerequisite or an implicit expectation. Notably, this paper does not aim or intend to explore the differences in salaries and job benefits between positions that require unwavering German language skills and positions that require little or no knowledge of the German language.  

Secondary research is a key component of any study, providing nearly unlimited insight into other institutions’ discoveries on a topic. Secondary research is an easy, cheap, and fast method of data collection that does not require “[involvement] in developing complicated data collection methods” (Nasrudin, 2025). Particularly with the rise of generative AI tools, such as ChatGPT, secondary research has become increasingly simple; it merely requires a quick prompt instructing an AI tool to search the internet for sources on any given topic. Additionally, secondary research is more varied, allowing a surplus of data to be used in comparison. However, secondary research comes with its respective disadvantages. For instance, the data may be inaccurate, as no piece can ever be fully verified as authentic. It may also be out-of-date, providing information that no longer applies to the current industry norms and requirements (Nasrudin, 2025).
