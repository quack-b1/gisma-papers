\chapter{Methodology}
To address the study’s research question, this paper employs a \emph{mixed-methods approach}, combining primary and secondary research strategies. This will provide a comprehensive perspective on how language impacts employability in Berlin, especially for individuals who are not fluent in German. Additionally, this mixed approach will enable a comparative analysis of secondary works and contemporary testimonials as of the summer of 2025.

\section{Primary Research Methods and their Limitations}
Primary data was collected via an online survey distributed through university mailing lists, online forums, and social media platforms. The survey was created using Microsoft Office Forms (see \hyperref[appendix:A]{Appendix~A}). It targeted expatriates and international students and was conducted anonymously with informed consent provided through a data protection disclaimer (datenschutzhinweis) written in both German and English. Additionally, the survey includes sections on demographics, academic background, employment preferences, language skills, and job search experience. Thus, the survey enables the collection of both statistical trends and subjective perceptions regarding the role of language in Berlin’s job market.

Primary research aids and benefits this paper by ensuring contemporary data is included. The survey used in this paper allows this research to “avoid unreliable information from outside sources” \citep{Indeed25}. However, primary research comes with its respective challenges \citep{k2024crucial}. The survey conducted in this study was particularly time-consuming and difficult to distribute. Issues arose when trying to convince individuals to take a few minutes to complete the survey, as many had refused to participate in such a study without an incentive. Also, the sample size was limited due to the difficulty in distributing the survey to the substantial number of Berlin residents. Some respondents may also downplay the issue of language requirements if they receive help from within their company, or, conversely, exaggerate the problem due to negative experiences during the hiring process.

\section{Secondary Research Methods and their Limitations}
Secondary data were collected through a content analysis of job listings on platforms such as LinkedIn, Indeed, StepStone, and businesses’ direct hiring portals. The listed jobs span various industries, including technology, marketing, education, healthcare, customer service, government, and finance. Each job listing was analyzed to determine the language in which it was posted, the language requirements listed, the industry, and the industry level. This data helps evaluate how frequently English-friendly positions are advertised, which industries offer such positions, and whether German language skills are a formal prerequisite or an implicit expectation. Notably, this paper does not aim to explore the differences in salaries and job benefits between positions that require proficiency in the German language and those that require little or no knowledge of the German language.

Secondary research is a crucial component of any study, providing valuable insights into the discoveries made by other institutions on a particular topic. It is an easy, cost-effective, and fast method of data collection that does not require “[involvement] in developing complicated data collection methods” \citep{Nasrudin25}. Especially with the rise of generative AI tools like ChatGPT, secondary research has become increasingly straightforward; it simply requires a quick prompt instructing an AI tool to search the internet for sources on any given topic. Moreover, secondary research is more diverse, allowing a wealth of data to be utilized for comparison. However, secondary research also has its drawbacks. For instance, the data may be inaccurate, as no piece can ever be fully verified as authentic. It may also be outdated, offering information that no longer aligns with current industry norms and requirements \citep{Nasrudin25}.

\section{Applications of Previous Research}
This study combines quantitative job listing analysis with qualitative surveys, informed by research on language in employment. It highlights German language requirements in job postings and their impact on job seekers, considering Germany's multilingual environment, where English is viewed as a corporate lingua franca alongside German. The survey explores language proficiency, job search strategies, and perceptions of Berlin's English friendliness, based on theories of linguistic integration by \citet{Shohamy06}, \citet{LinguaFranca}, and \citet{Neeley2012}. \citeauthor{gogolin02}'s (\citeyear{gogolin02}) insights on immigrants adopting the majority language also inform how this impacts Berlin's labor market. \par