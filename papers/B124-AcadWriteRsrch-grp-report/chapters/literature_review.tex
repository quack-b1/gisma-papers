\chapter{Literature Review}
\vspace{5pt}
\section{Berlin’s International Image}
Berlin has meticulously crafted a portrait of vibrancy and multicultural friendliness. As “Germany’s multicultural capital, [Berlin] is home to people from over 190 countries, creating a unique environment where many languages are spoken, and English is often understood” \citep{kummuni25Language}. Streamlined visa policies tailored for highly skilled workers, the proliferation of numerous start-up incubators offering many English-language networking opportunities, and a dynamically expanding tech ecosystem actively recruit a diverse international workforce. Berlin “even has an area called ‘Silicon Allee’” \citep{simplegermany25}, further reinforcing the city’s global image. \citet{bertelsmann} also notes that this multifaceted effort has been successful in attracting a diverse range of international talent, thereby contributing to the city’s cosmopolitan character.

However, despite the city’s immense efforts to appear English-friendly, language remains a significant barrier to employment. According to \citet{berbee24Mig}, proficiency in the local language has a direct and robust correlation to improved market outcomes, such as better wage levels and enhanced occupational attainment. “Improved German language skills effectively reduce the employment gap and the initial wage gap” for immigrants \citep{berbee24Mig}. This means that while English may unlock certain paths, a strong command of the German language is an indispensable qualification for a more entangled economic integration.

\section{Language and Employment in Globalized Cities}
One of the most consistent findings across multiple studies is the significant role that German language proficiency plays in influencing labor market outcomes for immigrants in Germany. German language proficiency is among the most crucial factors influencing employment and earnings among immigrants, especially for those outside the startup or academic sectors \citep{berbee24Mig}. The increasingly persuasive interconnectedness of a globalized world has immensely reshaped labor markets, labeling proficiency in the local language as a critical determinant for employment, especially for expatriates seeking employment \citep{Shohamy06}. The demand for multilingual employees has risen with the rise of globalization. This is because multinational corporations expand their reach to engage with a global clientele, thus creating a need for a widely accepted lingua franca \citep{LinguaFranca}. Consequently, the English language has indisputably emerged as the dominant “global language of business” \citep{Neeley2012}, serving as the primary medium for international communication.

\section{Previous Research on Language Requirements}
\citet{gogolin02} argues that European nations underestimate the actual linguistic diversity within their borders, thus neglecting the languages brought through immigrant integration. There is an “expectation that immigrants adapt to their new place of residence, also in the sense that they give up their inherited languages and ‘convert’ to the majority language” \citep{gogolin02}. This can further support Berbée’s and Stuhler’s claim that immigrants must learn the local language to integrate into the workforce. Gogolin highlights a hierarchy of language status in diverse contexts where German is the “dominant language.” To complement, \citet{Extra10diversity} highlights the importance of “[the] widely spoken home language” as an indicator of linguistic diversity. He argues that such linguistic diversity can often challenge monolingual mindsets and function as “agents of change” \citep{Extra10diversity}.

\section{Perception vs. Reality in Germany's Job Market}
Even in companies where English is the official corporate language, German frequently remains the primary language used for communication between German colleagues \citep{ErlingWalton07}. This creates a gap between the expectation of an English-only environment and the reality where German language skills are an absolute must-have characteristic. “In the past, companies conducted most of their business in German, and employed [few English-specialists] where they were most needed,” write \citet{ErlingWalton07}, noting that nowadays, “knowledge of English is necessary for communication with people from all over the world.”

Similarly, \citet{Extra10diversity} further suggests that while cities may be linguistically rich and informal in social settings, public institutions continue to operate in a monolingual manner. Formal and official communication from government entities, such as the Foreigners’ Office, Federal Employment Agency, and Tax Office, remains primarily conducted in German \citep{kummuni25Language}. Significant obstacles also exist for entry into non-tech or non-startup sectors outside of the comparatively sheltered and very English-friendly tech and startup bubble. A strong and frequently validated command of the German language is required by practically all traditional sectors, such as public services and regulated professions like human resources, accounting, law, and medicine \citep{simplegermany25}. “Jobs that require you to know or work with German laws are the least likely for you to get without speaking German” \citep{simplegermany25}. On the other hand, \citet{DamelangHaas12} highlight that “young foreigners and Germans face significantly lower barriers for employment entry in culturally more diverse German regions.” This perspective argues that international and diverse areas, such as Berlin, have “lower barriers for employment entry” \citep{DamelangHaas12}.

\section{Research Gap}
While literature addresses Berlin’s international appeal and the importance of local language proficiency for immigrants in Germany, a research gap persists in directly comparing Berlin’s perceived English-friendly job market with the actual linguistic requirements faced by non-German-speaking professionals across various sectors. Prior research has examined the gap between expectations and reality in German workplaces, where English is the official language. Still, German is often used, alongside the monolingual nature of public institutions. However, a comprehensive analysis quantifying this misalignment through job listing analysis and accounts from non-German-speaking job seekers in Berlin, beyond technology and startups, remains limited. This paper aims to fill this gap by providing insights into the challenges faced by this demographic within Berlin’s labor market.\par