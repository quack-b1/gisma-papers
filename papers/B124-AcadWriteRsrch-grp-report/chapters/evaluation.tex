\chapter{Results and Discussion}
\vspace{5pt}
\section{Job Listing Analysis}
An analysis of 191 Berlin job listings (see \hyperref[appendix:B]{Appendix~B}) revealed that 75.4\% specified listings required at least conversational German (19.9\% conversational, 55.5\% fluent), while 24.6\% required none. Fifty-two percent of listings were posted in English, 47.1\% in German, and 0.5\% in both. This contrasts with national data indicating that 97\% of jobs require German \citep{Jones25}. While the sample shows a greater presence of English, a significant German barrier persists, challenging Berlin’s perceived friendliness to English speakers.

\subsection{Information Technology}
Of 44 IT listings, 34.1\% needed no German, 25\% required conversational German, and 40.9\% needed fluent German. Sixty-five postings were in English. This sector is English-friendly, with many platforms listing English-speaking roles. Many tech companies foster “English-speaking environments,” creating an English-speaking “bubble”.   

\subsection{Finance}
Of 37 finance listings, 62.2\% required fluent German, 13.5\% conversational German, and 24.3\% no German, with 48.6\% being listed in German. This indicates that finance roles, particularly those involving German regulations, often require German fluency.

\subsection{Healthcare}
Core medical professions require German exams (B2-C2). Primary data shows that roles like `Medizinische Fachangestellte' require fluent German. Bureaucratic tasks are primarily in German. This represents a life-critical language barrier. 

\subsection{Education}
The English teaching market is peculiarly saturated. Certified teachers in public schools require proficiency in fluent German. Primary data shows mixed requirements (33.3\% none, 16.7\% conversational, 50\% fluent), with 70.8\% of listings posted in English.

\subsection{Public Services}
Of 24 public service listings, 58.3\% required fluent German, 12.5\% conversational German, and 29.1\% required none, with 50\% listed in English. Public service roles almost universally require proficiency in German due to the state administration and bureaucracy.   

\subsection{Manufacturing}
Of 23 manufacturing listings, 65.2\% required fluent German, 21.7\% required conversational German, and 13\% required no German proficiency. With 39.1\% of being listed in English and 4.3\% being listed in both languages, this suggests that blue-collar work, such as manufacturing, tends to require proficiency in German for ease of communication.

\subsection{Marketing}
Of 29 listings for marketing, 48.3\% required fluent German, 34.5\% required conversational German, and 17.2\% required none. Forty-eight percent of listings were in English. This suggests that German businesses tend to market primarily to German-speaking consumers, rather than international customers, given that the majority of positions require proficiency in the language.

\section{Survey Findings}
\vspace{5pt}
\subsection{Demographics Language Proficiency, and Perceptions of Respondents}
The survey gathered responses from young professionals and students, primarily aged 18-25 (69\%) and 26-35 (23\%). Most respondents held or were pursuing a bachelor’s degree (54\%), while the remainder held or were working towards a master’s degree (46\%). Many respondents were from Gisma University of Applied Sciences, reflecting the demographics of international students and expatriates. A significant majority (69\%) of participants reported Beginner (A1-A2) proficiency in German, with 15\% having no German skills. Additionally, 73\% were fluent in English, while the remaining 27\% possessed intermediate (B1-B2) English skills.

This linguistic profile, which indicates a highly educated group with strong English but limited German, supports the argument that Berlin attracts English-speaking professionals who are underprepared for the broader German-speaking labor market. Respondents indicated that only a tiny number of job listings did not require German language skills. Sixty-two percent reported that only 0-25\% of job listings in their desired field were posted in English, with 19\% reporting 26-60\%. In further support of the assertion that German is almost always required in the workplace, 42\% stated that German was always required in their field, while 38\% said it was often necessary. This inconsistency between Berlin’s English-friendly reputation and its actual job market outlines the widespread reality for job seekers, creating an imminent employment challenge.


\subsection{Impact of German Language Skills on Job Searches}
Sixty-nine percent of participants reported job application rejections due to insufficient German skills. Only 12\% were able to secure employment without fluency in German, primarily in the hospitality and information technology sectors. Additionally, 31\% were not invited to any interviews, suggesting that language barriers hinder initial screening and indicating limited English-only opportunities. This empirical evidence supports the claim that language proficiency has a significant impact on employment.

Before relocating to Berlin, 50\% of participants believed they could find a job there without speaking German, which aligns with the city’s image. However, when asked if Berlin’s English-friendly reputation accurately reflects the job market, 50\% stated that this was not true at all, while an additional 23\% expressed that it was only slightly true. This stark contrast reveals a significant gap between reputation and reality. Berlin’s marketing creates unrealistic expectations that lead to frustration and dissatisfaction for expatriates and international students.

\section{Analysis}
The study reveals a significant gap between Berlin’s image as an English-friendly city and its labor market realities. Survey respondents report a shortage of English-only job listings (0-25\% for 62\%), aligning with national data indicating that 96-97\% of postings require German proficiency. An 88\% rejection rate due to inadequate German skills highlights the linguistic demands in most sectors. The success rate for non-fluent German speakers demonstrates a labor market where English-only positions are rare. While English suffices in tourism, it is insufficient in traditional industries and public services, which are primarily conducted in German. Despite Berlin’s marketing as an English-friendly city, actual labor market conditions require German proficiency, posing challenges for non-German speakers. A lack of German skills hinders career advancement and income growth. Proficiency is crucial for navigating bureaucracy and internal communication. New residency regulations mandating German proficiency formalize this situation, acting as a socio-economic gatekeeper and relegating non-German speakers to a secondary labor market with limited integration opportunities.

Berlin’s competitive market faces an oversaturation of English-only job openings. Recruiters often seek a “perfect fit,” including one that matches their language preferences. Daily life and administrative tasks remain predominantly German-speaking. Implicit German expectations can lead to rejections, and employers may exhibit statistical discrimination. This often limits qualified international professionals to low-skilled roles, leading to underemployment.

The tech and startup ecosystem offers the best path for English-speaking professionals, particularly in software development and AI. Learning German is vital as it broadens job opportunities and enhances career prospects. Government language courses boost employment opportunities. Berlin presents an English-friendly image but reveals a German-dominated reality, requiring job seekers to adapt their strategies accordingly. Opportunities lie in competitive niches, while the broader market remains inaccessible without German. The focus should be on long-term German language investment and targeted networking to break through barriers and achieve integration.
