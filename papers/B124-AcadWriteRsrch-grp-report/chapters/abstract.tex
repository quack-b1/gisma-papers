\pdfbookmark[1]{Abstract}{abstract}
\chapter*{Abstract} 
\phantomsection
\noindent\setstretch{1.2}This study examines the disparity between Berlin’s well-established international reputation as an English-friendly city and the actual language requirements that non-German-speaking professionals encounter in the labor market. By employing a mixed-methods approach, the research combines an analysis of current job listings from various industries with primary survey data collected from international students and expatriates residing in the Berlin Metropolitan Area. The findings reveal a significant discrepancy: although the city is perceived as an accessible hub for English speakers, most full-time job listings still require proficiency in German. The survey results reinforce this observation, indicating that many non-German-speaking professionals experience rejection due to insufficient German language skills. At the same time, the perception of an English-only job market is considerably higher than what is observed. The study concludes by emphasizing that having German language skills remains a significant hurdle to employment for non-German speakers in Berlin. It advocates for a more accurate representation of the city’s language job market requirements to help manage expectations and support improved professional integration.